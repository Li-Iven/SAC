\documentclass{article}
\usepackage{sagetex}
\setlength{\sagetexindent}{10ex}
\usepackage{amsmath}
\usepackage[russian]{babel}
\usepackage{hyperref}
\usepackage[left=15mm,top=30mm,bottom=30mm,right=15mm]{geometry}
\begin{document}
\begin{center}
\Large{\textbf{Лабораторная работа 2. }}  

\Large{\textbf{Задание 3 - приведение поверхности второго порядка к каноническому виду.}}
\end{center}

\section*{Задание}
\begin{enumerate}
\item Для заданной уравнением фигуры: упростить, привести к каноническому виду.
\item Построить исходную фигуру и упрощенную.
\item Собственные вектора и числа получать вручную, сравнить с результатом встроенных функций.
\end{enumerate}

\begin{center}
Вариант 1

$f = 7*x^2 + 8*x*y + 3*y^2 + 8*x*z + 6*y*z + 3*z^2 + 6*x + y + 7$
\end{center}

\section*{Построение исходной поверхности}
Функция:

\begin{sagesilent}
var("x y z")
\end{sagesilent}
\begin{sageblock}
f(x, y, z) = 7*x**2 + 8*x*y + 3*y**2 + 8*x*z + 6*y*z + 3*z**2 + 6*x + y + 7
\end{sageblock}
Выведем ее:
\begin{center}
$\sage{f(x=x, y=y, z=z)}$
\end{center}
Построим ее график:
\begin{center}
\sageplot{implicit_plot3d(f, (x, -30, 10), (y, -30, 10), (z, -10, 30), color = 'green', figsize = 3)}
\end{center}
\newpage
\section*{Приведем поверхность к каноническому виду}

Для приведения поверхности к каноническому виду надо сначала вычислить ортогональные инварианты.

Пусть общее уравнение поверхности второго порядка будет выгялдеть так:


$$a_{11}*x^2 + a_{22}*y^2 + a_{33}*z^2 + 2*a_{12}*xy  + 2*a_{13}*xz + 2*a{23}*yz  + 2*a_{1}*x + 2*a_{2}*y + 2*a_{3}*z + a_{0} = 0$$

Тогда:

$$\tau_{1} = a_{11} + a_{22} + a_{33}$$

$$\tau_{2} = \begin{bmatrix}
    a_{11} & a_{12} \\
    a_{12} & a_{22} \\
\end{bmatrix} + 
\begin{vmatrix}
    a_{11} & a_{13}\\
    a_{13} & a_{33}\\
\end{vmatrix} + 
\begin{vmatrix}
    a_{22} & a_{23}\\
    a_{23} & a_{33}\\
\end{vmatrix}$$

A = $\begin{pmatrix}
    a_{11} & a_{12} & a_{13}\\
    a_{12} & a_{22} & a_{23}\\
    a_{13} & a_{23} & a_{33}\\
\end{pmatrix}$
$\qquad$
delta = $det(A)$ = $\begin{vmatrix}
    a_{11} & a_{12} & a_{13}\\
    a_{12} & a_{22} & a_{23}\\
    a_{13} & a_{23} & a_{33}\\
\end{vmatrix}$
$\qquad$
Delta = $det(B)$ = $\begin{vmatrix}
    a_{11} & a_{12} & a_{13} & a_{1}\\
    a_{12} & a_{22} & a_{23} & a_{2}\\
    a_{13} & a_{23} & a_{33} & a_{3}\\
    a_{1} & a_{2}  & a_{3} &  a_{0}
\end{vmatrix}$


Составим матрицу А для квадратичной формы и матрицу B, состоящую из коэффициентов квадратичной формы, линейной формы и свободного члена $a_{0}$:

\begin{sageblock}
A = matrix([
    [7, 4, 4],
    [4, 3, 3],
    [4, 3, 3]
])
B = matrix([
    [7, 4, 4, 3],
    [4, 3, 3, 0.5],
    [4, 3, 3, 0],
    [3, 0.5, 0, 7]
])
\end{sageblock}

Найдем инварианты:

\begin{sageblock}
tau1 = A.trace()
tau2 = A[0:2, 0:2].det() + A[[0, 2], [0, 2]].det() + A[1:3, 1:3].det()
delta = A.det()
Delta = B.det()
\end{sageblock}

Получили следующие результаты:

\begin{center}
$$tau1 = \sage{tau1}$$
$$tau2 = \sage{tau2}$$
$$delta = \sage{delta}$$
$$Delta = \sage{Delta}$$
\end{center}

По вычисленным ортогональным инвариантам определяем \textbf{тип поверхности - эллиптический параболоид}.

У элептического параболоида каноническое уравнение выглядит следующим образом:

\begin{center}
$\Large \frac{x^2}{a^2} + \frac{y^2}{b^2} = 2*z$
\end{center}

\begin{sagesilent}
var('l')
\end{sagesilent}

\begin{sageblock}
lam(l) = A.charpoly(var='l')
roots = []
for i in solve(lam(l)==0, l):
    roots.append(i.rhs())
\end{sageblock}

\begin{center}
$$\lambda_1 = \sage{roots[0]}$$
$$\lambda_2 = \sage{roots[1]}$$
$$\lambda_3 = \sage{roots[2]}$$

В численном виде: 

$$\lambda_1 = \sage{roots[0].n()}$$
$$\lambda_2 = \sage{roots[1].n()}$$
$$\lambda_3 = \sage{roots[2].n()}$$

\end{center}

Теперь найдем собственные значения через встроенную функцию.

\begin{sageblock}
A.eigenvalues()
\end{sageblock}

\begin{sagesilent}
lam_mech = A.eigenvalues()
\end{sagesilent}

Получили: 
$$\sage{lam_mech[0]}$$
$$\sage{lam_mech[1]}$$
$$\sage{lam_mech[2]}$$ 

Результаты вычисления собственных значений совпали.

Теперь вычисляем коэффициенты канонического уравнения:

\begin{sagesilent}
var("x1 y1 z1")
\end{sagesilent}

\begin{sageblock}
a = n(sqrt(-Delta/(roots[0]**2 * tau2)))
b = n(sqrt(-Delta/(roots[1]**2 * tau2)))
\end{sageblock}

\begin{sageblock}
canonical(x1, y1, z1) = x1**2/a + y1**2/b - 2*z1
\end{sageblock}

$$\sage{canonical(x1, y1, z1)}$$

Построим график упрощённой функции:

\begin{center}
\sageplot{implicit_plot3d(canonical, (x1, -30, 10), (y1, -30, 10), (z1, -10, 30), color = 'orange', figsize = 3)}
\end{center}

\end{document}